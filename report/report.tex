\documentclass{article}
\usepackage[utf8]{inputenc}
\usepackage[brazil]{babel}
\usepackage[colorlinks, linkcolor=blue, urlcolor=blue, citecolor=blue]{hyperref}
\usepackage[a4paper, left=20mm, right=20mm, top=20mm, bottom=20mm]{geometry}

\title{\textbf{Relógio-ponto em blockchain \\
        \large INE5429 - Segurança em Computação}}
\author{
    Caique Rodrigues Marques \\
    {\texttt{c.r.marques@grad.ufsc.br}}
    \and
    Fernando Jorge Mota \\
    {\texttt{contato@fjorgemota.com}}
    \and
    Lucas Finger Roman \\
    {\texttt{lfrfinger@gmail.com}}
}
\date{}

\begin{document}

\maketitle
% \tebleofcontents
\section{Requisitos funcionais}
\begin{itemize}
    \item Lançamento de \textit{smart contract} contendo informações do contrato do
      trabalhador;
    \item Lançamento de \textit{smart contract} em inicio de expediente do funcionário;
    \item Lançamento de \textit{smart contract} em final de expediente do funcionário;
    \item Lançamento de \textit{smart contract} de alteração de contrato de trabalho;
    \item Lançamento de \textit{smart contract} de finalização do contrato de trabalho;
    \item Lançamento de \textit{smart contract} de valor de hora extra;
\end{itemize}

\subsection{\textit{Smart contract} contendo informações do contrato do trabalhador}
O sistema deve ser capaz de lançar um \textit{smart contract} contendo
informações sobre o contrato do trabalhador com a empresa. O contrato deve
conter informações tais como data de emissão, salário do funcionário, jornada
mensal de trabalho, função, férias e
circunstâncias especiais.

\subsection{\textit{Smart contract} em inicio de expediente do funcionário}
O sistema deve ser capaz de lançar um \textit{smart contract} contendo as
informações relacionadas ao início do expediente por um trabalhador, de forma a
ser possível computar a quantidade de horas trabalhadas.

\subsection{\textit{Smart contract} em final de expediente do funcionário}
O sistema deve ser capaz de lançar um \textit{smart contract} de final do
experiemente do funcionário, podendo assim registrar a quantidade de horas trabalhadas.

\subsection{\textit{Smart contract} de alteração de contrato de trabalho}
O sistema deve ser capaz de mudar o contrato do trabalhador enquanto este
ainda trabalhar para a empresa. Para tanto, deve ser capaz mudar informações
pertinentes ao trabalho, tais como a jornada mensal de trabalho, a duração de férias,
a função do funcionário e as circunstâncias especiais. Não deve ser possível modificar
informações tais como o CPF do funcionário, a data de nascimento de funcionário
nem demais informações que não fazem parte do acordo entre o empregado e o
empregador. Além disso, eventuais modificações devem ter suas datas anotadas.

\subsection{\textit{Smart contract} de finalização do contrato de trabalho}
O sistema deve possuir a funcionalidade de finalizar o contrato de um empregado
com seu empregador, seja por demissão por justa causa ou por outro motivo.
Porém, como este tipo de informação serve para denegrir a imagem do trabalhador,
o motivo da revogação do contrato não deve estar listada no \textit{smart contract}.

\subsection{\textit{Smart contract} de valor de hora extra}
O sistema deve possuir o valor a ser pago por hora extra do funcionário com base em
um percentual de seu salário base por hora. Desta forma, tendo em mãos a quantidade de horas
trabalhadas pelo funcionário e comparando com sua jornada de trabalho, é possível calcular
o valor a ser recebido de hora extra.


\section{Requisitos não-funcionais}
\begin{itemize}
    \item Usar Ethereum como plataforma de \textit{blockchain};
    \item Usar Solidity como linguagem de programação.
\end{itemize}

\subsection{Usar Ethereum com plataforma de \textit{blockchain}}
Ethereum é a plataforma de computação distribuída baseada em \textit{blockchain} com integração de \textit{smart contracts}, que pode ser escrito em variadas
linguagens de programação.

\subsection{Usar Solidity como linguagem de programação}
Será usada a linguagem \textit{Solidity}, que é orientada a contratos desenvolvida para
escrever \textit{smart contracts} e que suporta variadas implementações de \textit{blockchain}, incluindo a o próprio Ethereum.

\section{Critérios de sucesso}
\begin{itemize}
    \item Implementação de pelo menos 5 dos 6 requisitos funcionais definidos;
    \item Conseguir computar valor a ser pago ao trabalhador a partir das horas
      trabalhadas e seus respectivos tipos, usando-se das informações presentes
      no contrato de trabalho e de inicio e fim do expediente;
    \item Computar atributos derivados diretamente dos contratos;
\end{itemize}
\end{document}
