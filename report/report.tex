\documentclass{article}
\usepackage[utf8]{inputenc}
\usepackage[brazil]{babel}
\usepackage[colorlinks, linkcolor=blue, urlcolor=blue, citecolor=blue]{hyperref}
\usepackage[a4paper, left=20mm, right=20mm, top=20mm, bottom=20mm]{geometry}

\title{\textbf{Relógio-ponto em blockchain \\
        \large INE5429 - Segurança em Computação}}
\author{
    Caique Rodrigues Marques \\
    {\texttt{c.r.marques@grad.ufsc.br}}
    \and
    Fernando Jorge Mota \\
    {\texttt{contato@fjorgemota.com}}
    \and
    Lucas Finger Roman \\
    {\texttt{lfrfinger@gmail.com}}
}
\date{}

\begin{document}

\maketitle
\tableofcontents{}
\section{Requisitos funcionais}

\subsection{Permitir a inserção informações do contrato do trabalhador pelo chefe}
O sistema deve ser capaz de lançar um \textit{smart contract} contendo
informações sobre o contrato do trabalhador com a empresa. O contrato deve
conter informações tais como data de emissão, salário do funcionário, jornada
mensal de trabalho, função, férias e circunstâncias especiais.

\subsection{Permitir a entrada de inicio/final do expediente pelo funcionário}
O sistema deve ser capaz de lançar um \textit{smart contract} contendo as
informações relacionadas ao início e ao final do expediente de um trabalhador,
de forma que seja possível computar as horas trabalhadas.

\subsection{Permitir a alteração de contrato de trabalho do funcionário pelo chefe}
O sistema deve ser capaz de mudar o contrato do trabalhador enquanto este
ainda trabalhar para a empresa. Para tanto, deve ser capaz mudar informações
pertinentes ao trabalho, tais como a jornada mensal de trabalho, a duração de férias,
a função do funcionário e as circunstâncias especiais. Não deve ser possível modificar
informações tais como o CPF do funcionário, a data de nascimento de funcionário
nem demais informações que não fazem parte do acordo entre o empregado e o
empregador. Além disso, eventuais modificações devem ter suas datas anotadas.

\subsection{Realizar ao chefe realizar a finalização do contrato de trabalho do funcionário}
O sistema deve possuir a funcionalidade de finalizar o contrato de um empregado
com seu empregador, seja por demissão por justa causa ou por outro motivo.
Porém, como este tipo de informação serve para denegrir a imagem do trabalhador,
o motivo da revogação do contrato não deve estar listada no \textit{smart contract}.

\subsection{Permitir ao chefe realizar a entrada de jornada de trabalho do funcionário}
O sistema deve ser capaz de multiplicar o número de horas trabalhadas, caso elas
ocorram em determinados períodos do dia ou caso o funcionário trabalhe em horas que exijam o pagamento de valores adicinais.
Por exemplo, pode ser considerado que um trabalho realizado durante a madrugada
tenha 1.2 vezes o valor de um trabalho realizado durante o período da tarde e que
hora extra tenha um valor multiplicativo de 1.5 vezes em seu valor, de acordo com o definido pela lei trabalhista.

\section{Requisitos não-funcionais}

\subsection{Usar Ethereum com plataforma de \textit{blockchain}}
Ethereum é a plataforma de computação distribuída baseada em \textit{blockchain} com integração de \textit{smart contracts}, que pode ser escrito em variadas
linguagens de programação.

\subsection{Usar Solidity como linguagem de programação}
Será usada a linguagem \textit{Solidity}, que é orientada a contratos desenvolvida para
escrever \textit{smart contracts} e que suporta variadas implementações de \textit{blockchain}, incluindo a o próprio Ethereum.

\subsection{Usar dados de autenticação das operações para checar se a operação é válida}
O sistema deve garantir que quem está realizando determinada operação \textbf{possa} efetuá-la de fato. Por exemplo, funcionário não pode alterar seus dados do contrato de trabalho, funcionário não pode registrar entrada e saída de outro funcionário, e uma pessoa não pode registrar mudanças sem ser funcionário da empresa.

\section{Dados Externos}
Dados externos são as informações que são necessárias para o projeto, portanto, os dados de entrada. Em um \textit{smart contract} ele pode conter as informações a seguir, que depende de qual o tipo de \textit{smart contract} tratado:
\begin{itemize}
    \item Salário do funcionário;
    \item Função do funcionário;
    \item Nome do funcionário;
    \item Jornada de trabalho do funcionário;
    \item Férias do funcionário;
    \item Circunstâncias especiais do funcionário;
    \item Horário de início do expediente;
    \item Horário de final do expediente.
\end{itemize}

\section{Critérios de sucesso}
Para ser considerado um sucesso, o projeto deve ter critérios bem especificados e, em especial, precisa cumprí-los. No caso deste projeto, selecionamos os seguintes critérios:

\begin{itemize}
    \item Implementação dos \textit{smart contracts} necessários para a realização dos requisitos funcionais especificados;
    \item Presença de comentários documentando o código criado para o projeto - para fins de manutenção posterior;
    \item Conseguir computar valor a ser pago ao trabalhador a partir das horas trabalhadas e seus respectivos multiplicadores, usando-se das informações presentes no contrato de trabalho e nos dados de inicio e fim do expediente do funcionário;
    \item Conseguir obter registro de horas trabalhadas de cada funcionário;
    \item Computar atributos derivados diretamente dos contratos.
\end{itemize}
\end{document}
